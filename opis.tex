\documentclass[a4paper]{article}
\usepackage[utf8]{inputenc}
\usepackage[T1]{fontenc}
\usepackage[slovene]{babel}
\usepackage{lmodern} 
\usepackage{hyperref}
\usepackage{blindtext}
\usepackage{amsmath} 
\usepackage{amsthm}   
\usepackage{amssymb} 

\title{Poti v kvadratu \\
       \large Opis projekta}
\author{Sara Papež, Tim Kalan}

\begin{document}
\begin{titlepage}
 \maketitle

\end{titlepage}

\section{Opis problema}
V projektu si bomo ogledali naslednjo situacijo: V enotskem kvadratu si zamislimo 
množico $P$, ki vsebuje $n$ točk. To množico nato proglasimo za množico vozlišč 
grafa in povežemo vse točke, ki so oddaljene manj od vnaprej podanega parametra $r$.
Nato pridobimo drevo najkrajših poti od naključno izbrane točke - \textit{korena}.

Jedro projekta leži v spreminjanju parametra $r$ in števila točk $n$ ter opazovanju 
dogajanja. Zanimalo nas bo recimo, kako se spreminja dolžina drevesa najkrajših poti,
kako se spreminja vsota dolžin od korena do ostalih, kako se spreminja dolžina 
najkrajše in najdaljše poti, ... 
% tu lahko dodava še poljubna vprašanja


\section{Programsko okolje in implementacija}
Jedro naloge bo vsebovano v razredu Kvadrat, implementiranemu v programskemu jeziku
\emph{Python}. Zaenkrat načrtujemo, da bomo to jedro potem uvozili v \emph{Jupyter
notebook} in se tam malo ">igrali"< s parametri in raznimi vizualizacijami.

\section{Načrt za delo}
Zaenkrat je načrtovana implementacija \emph{Dijkstrovega algoritma} za iskanje 
drevesa najkrajših poti - če bo le šlo, tudi hitrejšo implementacijo s časovno 
zahtevnostjo $\mathcal{O}(m + n\log n)$ (kjer je $m$ število povezav) in uporabo 
prednostne vrste.

Možne posplošitve, ki jih vidimo so v tri dimenzije oz. $n$ dimenzij (kjer vizualni
aspekt seveda odpade).

\end{document}}