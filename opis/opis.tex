\documentclass[12pt,a4paper]{article}
\usepackage[utf8]{inputenc}
\usepackage[T1]{fontenc}
\usepackage[slovene]{babel}
\usepackage{lmodern} 
\usepackage{hyperref}
\usepackage{amsmath} 
\usepackage{amsthm}   
\usepackage{amssymb} 

\title{Poti v kvadratu \\
       \large Opis projekta}
\author{Sara Papež, Tim Kalan}
\date{20.~november 2020}

\begin{document}
\begin{titlepage}
 \maketitle

\end{titlepage}

\section{Opis problema}
V projektu si bomo ogledali naslednjo situacijo: V enotskem kvadratu naključno 
generiramo $n$ točk in jih zberemo v množico $P$. To množico nato proglasimo 
za množico vozlišč $V$ grafa $G$ in povežemo vse točke, ki so ena od druge oddaljene 
manj od vnaprej podanega parametra največje dovoljene razdalje $r$. To nam da torej 
graf $G = (V, E)$, kjer je $E$ množica povezav, ki je odvisna od paramtera $r$. V 
nadaljevanju se osredotočimo na računanje drevesa najkrajših poti od naključno 
izbrane točke - \textit{korena}.

Jedro projekta leži v spreminjanju parametra $r$ in števila točk $n$ ter opazovanju 
dogajanja. Zanimalo nas bo recimo:

\begin{itemize}
       \item Kako se spreminja dolžina drevesa najkrajših poti?
       \item Kako se spreminja vsota dolžin od korena do ostalih?
       \item Kako se spreminja dolžina najkrajše in najdaljše poti?
       \item Kater parameter povzroči bolj drastične spremembe?
       \item ...
\end{itemize}


\section{Programsko okolje in implementacija}
Odločili smo se za implementacijo v programskem jeziku \emph{Python}, saj smo z njim
najbolje seznanjeni. Zaenkrat je ideja, da v eno skripto napišemo celotne ">možgane"<
oz. model: najverjetneje bo to nek razdred Kvadrat, ki bo vseboval generator naključnih 
točk, generator bližnjih (glede na $r$) točk, algoritem za računanje drevesa nakjrajših 
poti in pa razne metode za vizualizacijo dobljenih rezultatov. Vizualizacija bo izvedena 
s pomočjo paketa \emph{Matplotlib}.

To jedro naloge bi potem uvozili v \emph{Jupyter notebook} in v tem okolju dejansko 
generirali točke, jih vizualizirali, spreminjali parametre in opazovali kaj se dogaja
z odgovori na zgornja vprašanja.


\section{Načrt za delo}
Zaenkrat je načrtovana implementacija \emph{Dijkstrovega algoritma} za iskanje 
drevesa najkrajših poti - če bo le šlo, tudi hitrejšo implementacijo s časovno 
zahtevnostjo $\mathcal{O}(m + n\log n)$ (kjer je $m$ število povezav) in uporabo 
prednostne vrste. Zanj smo se odločili, saj gre za utežen graf, ki ima 
nenegativne uteži, različne od ena (zato odpade \emph{iskanje v širino}). 

Poleg spreminjanja parametrov, nas zanima tudi, kaj se dogaja z razdaljami/dolžinami 
dreves, če poženemo simulacijo večkrat z enakimi parametri. Poleg tega bi bilo zanimivo 
videti, kako se spreminja čas izvjanja, ko večamo/manjšamo parametre. 

Možne posplošitve, ki jih zaenkrat vidimo so recimo premik v tri dimenzije; torej bi 
opazovali problem na enotski kocki. Če se izkaže primerjava za zanimivo, bi lahko 
problem prenesli tudi v $n$ dimenzij, tu bi seveda opustili vizualizacijo. Morda bi 
bilo zanimivo videti tudi, kaj se zgodi ob generiranju točk glede na kakšno drugo
porazdelitev, kot enakomerno.

\end{document}}